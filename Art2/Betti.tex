\documentclass[a4paper,12pt]{article}
\usepackage{glaudo_en_pkg}

\theoremstyle{remark}
\newtheorem{note}[results]{Note}


\newcommand{\bigO}{\bigo}

\DeclareMathOperator{\ord}{ord}



\begin{document}

\paragraph{Canonical height and Betti map}

Throughout this section, $\pi:E \rightarrow \P_1$ denotes the Legendre scheme, i.e. the elliptic surface associated to the following elliptic curve, defined over the function field $\C(\lambda)$ of $\P_1$:
\[
y^2=x(x-1)(x-\lambda).
\]

Let $S=\{0,1,\infty\}$ be the set of points of bad reduction for $\pi$. In the future we will also use the notation $\mathcal{S}=\pi^{-1}(S)$. 
%Let $D \subset \P_1(\C)\setminus S$ denote a simply connected domain. Since $D$ is simply connected, we can choose a local pair of periods for the elliptic integral, and denote them by $(\rho_1(\lambda),\rho_2(\lambda))$, for $\lambda \in D$.

	In this note, we will frequently work with the abelian logarithm of an elliptic curve $X/\C$. We recall that, for this to be well defined, a lattice $\Lambda_X \subset \C$, such that $\C/\Lambda_X \cong X$ has to be chosen. 
	
	\begin{remark}\label{Rmk:weierstrass}
			When a Weierstrass form $y^2=x^3+Ax+B$ for $X$ is chosen, there exists one unique such lattice $\Lambda_X$ that satisfies the extra condition $g_2(\Lambda_X)=-4A, g_3(\Lambda_X)=-4B$. When we work with a specific Weierstrass equation (such as, for instance, the Legendre form), we assume that this canonical choice of a lattice for the abelian logarithm has been made.
	\end{remark}
	


\begin{definition}\label{bettidef}
Let $\pi:E \rightarrow \P_1$ be the Legendre scheme, and let $D \subset \P_1 \setminus S$ be a simply connected domain. Let $(\rho_1(\lambda),\rho_2(\lambda)), \ \lambda \in D$, be a holomorphic choice of periods for the elliptic logarithm. For any $P \in \pi^{-1}(D)$, we call the \textit{Betti coordinates} of $P$, and denote them by $(\beta_1,\beta_2) \in (\R/\Z)^2$, the unique elements of $\R/\Z$, such that the following equality holds:
\[
\log^{ab}(z)=\beta_1\rho_1(\pi(z))+\beta_2\rho_2(\pi(z)),
\]
where $\log^{ab}$ denotes the abelian logarithm multivalued function. We call the \textit{Betti map} the map that associates to a point $P$ its Betti coordinates.
\end{definition}

\begin{note}
	We note that, although for the Betti map $(\beta_1,\beta_2)$ to be well-defined, one needs to restrict oneself to a simply connected fundamental domain, the $1$-forms $\de \beta_1$, $\de \beta_2$, which are going to appear commonly in this paper, are well-defined on $\pi^{-1}(D)$, for any simply connected open $D \subset \P_1 \setminus S$. Moreover, the $2$-form $\de \beta_1 \wedge \de \beta_2$ is well-defined on $E \setminus \mathcal{S}$, i.e. it is independent of the local choice of periods.
\end{note}

If we happen to have a specific section $\sigma:B \rightarrow E_B$, where $\pi_B:E_B \rightarrow B$ is a base changed model of $E$, we sometimes call, with a slight abuse of notation, the Betti coordinates of a point $P \in B$, the Betti coordinates of the point $\sigma(P) \in E_B$ (after a specific choice of a simply connected domain $D \subset B$ has been made).
 
For a more detailed exposition about the Betti map, see e.g. \cite[Section 1.1]{bettitorsion}.
\begin{theorem}\label{Prop:equality}
	Let $r:B \rightarrow \P_1$ be a finite morphism, and $B$ be a smooth complete complex curve. Let $\sigma:B \rightarrow E_B \defeq E\times_{(\pi,r)}B$ be an algebraic section of $\pi_B:E_B \rightarrow B$.  Then, the following equality holds:
	\begin{equation}\label{soughtequality}
	\hat{h}(\sigma) = \int_{B \setminus r^{-1}(S)} \sigma^*(\de \beta_1 \wedge \de \beta_2),
	\end{equation}
	where $(\beta_1,\beta_2)$ is a (local branch) of the Betti map on $E$. 
	%We note that, although for the Betti map $(\beta_1,\beta_2)$ to be well-defined, one needs to restrict oneself to a simply connected fundamental domain, the $(1,1)$-form $\de \beta_1 \wedge \de \beta_2$ is well-defined on $E_B \setminus r^{-1}(S)$. 
\end{theorem}

Before coming to the proof, we recall the following Lemma, which follows directly from \cite[Theorem 1.3]{ominimal}:
\begin{lemma}\label{ominimalmeasure}
	Let $X \subset \R^k$ be a bounded definable subset of $\R^k$ in an o-minimal structure. Let, for each $n>0$,
	\[
	\alpha_n \defeq \#\left\{\left(\frac{a_1}{n},\dots,\frac{a_k}{n} \right)\in X: \ a_1,\dots, a_k \in \Z \right\}.
	\]
	Then, the limit $\lim_{n \to \infty} n^{-k}\alpha_n$ exists and is equal to $\mu(X)$, the Lebesgue measure of $X$.
\end{lemma}


\begin{proof}[Proof of Theorem \ref{Prop:equality}]
	Let us first consider the case where $\sigma$ is torsion. Since the Betti map of a torsion section is constant, in this case, left and right hand side of (\ref{soughtequality}) are both equal to $0$. 
	
	We restrict now to the case where $\sigma$ is not torsion, and consider, for each $n \geq 1$, the following quantity:
	\begin{equation}\label{targetorsion}
		A_n \defeq \# \{t \in B(\C)\setminus r^{-1}(S): \sigma(t) \text{ is }n-\text{torsion in } E_t\}.
	\end{equation}
	We notice that, since $\sigma$ is not torsion, this quantity is finite for each $n \geq 1$.
	We claim the following (which obviously implies the thesis):
	\begin{enumerate}
		\item The limit $\lim_{n \to \infty}\frac{A_n}{n^2}$ exists, is finite, and $\lim_{n \to \infty}\frac{A_n}{n^2}=\hat{h}(\sigma)$;
		\item The limit $\lim_{n \to \infty}\frac{A_n}{n^2}=\int_{B \setminus r^{-1}(S)} \sigma^*(\de \beta_1 \wedge \de \beta_2)$.
	\end{enumerate}
	Let us first prove $(1)$. We know that (see e.g. \cite[Section III.9]{silverman1994advanced} or \cite[Sections 2,3]{SerreMW}):
	\[
	\hat{h}(\sigma) =\lim_{n \to \infty} \frac{<n \sigma, O>}{n^2},
	\] 
	where $n\sigma$ denotes, with a slight abuse of notation, the graph of the section of $n \sigma$, $O$ denotes the zero section of $\pi_B$, and $<n \sigma, O>$ denotes the intersection product in a smooth proper model of $E_B$. We write:
	\[
	<n \sigma, O>=A_n+\delta_n+s_n,
	\]
	where $s_n$ denotes the intersection of $n \sigma$ and $O$ on the singular fibers of $\pi_B:E_B \rightarrow B$, and $\delta_n$ is a correction term that keeps track of the intersection that happens with multiplicity greater than $1$. I.e.:
	\[
	\delta_n=\sum_{t \in B\setminus r^{-1}(S)} (\ord_t(n\sigma)-1),
	\]
	where $\ord_t(n\sigma)$ denotes the multiplicity of intersection of $n\sigma$ and $O$ at $t$. 
	We will now prove that $\delta_n+s_n=O(1)$, as $n \to \infty$. 
	
	We prove now that $s_n=O(1)$. To do so, it suffices to show that, for a point $s \in B$ of bad reduction for $\pi_B$, $<n\sigma,O>_s$ (i.e. the local multiplicity of intersection) is bounded in $n>0$. Let $s \in r^{-1}(S)$, and let $m \in \Z$ be the least positive integer such that $(m\sigma)(s)=0$ on the singular fiber $E_s$. Since $<n \sigma, O>_s$ can be positive only when $n$ is a multiple of $m$, we may assume without loss of generality (replacing, if necessary, $\sigma$ by $m\sigma$), that $m=1$, i.e. that $\sigma(s)=0$. 
	
	Let now $\lambda$ denote a local parameter for $s \in B$. Let $K$ denote the completed field of Laurent series $\C\{\{\lambda\}\}$, and let $O_K$ denote the ring of integers of $K$. Let:
	\[
	E_1(K)\defeq \{P \in E(K):P(s)=0\}.
	\]
	
	We recall that $E_1(K)$ has the structure of a Lie group over the local field $K$, given by the restriction of the sum operation on the elliptic curve $E/K$. The Lie group $E_1(K)$ is isomorphic to a Lie group $(\lambda O_K,\tilde{+})$, in a way that we briefly recall now.
	
	%The Lie group $E_1(K)$ is isomorphic to the Lie group $(\lambda O_K,\tilde{+})$, where the isomorphism associates to a point $P$ its $z$-coordinate $z(P)\defeq x(P)/y(P)$. Here $\tilde{+}:$ is an operation defined through the formal group structure of the elliptic curve $E/K$. 
	We recall from \cite[Chapter IV]{silverman} that the formal group on the elliptic curve $E/K$ is the unique formal power series:
	\[
	F(+)(X,Y) \in K[[z_1,z_2]],
	\]
	such that
	\[
	F(+)(z(P),z(Q))=z(P+Q), \quad P, Q \in E
	\]
	as \textit{formal} power series, where $z(P)\defeq x(P)/y(P)$, denotes the $z$-coordinate in a given minimal Weierstrass model for $E/K$. 
	
	We recall, moreover, that the following holds:
	\[
	F(+)(z_1,z_2)=z_1+z_2+O(z^2).
	\]
	
	One can verify that, when an integral Weierstrass model is chosen, $F(+) \in O_K[[z_1,z_2]]$. Hence, $F(+)(X,Y)$ converges when $X, Y \in \lambda O_K$. One can then see that the Lie group $E_1(K)$ is isomorphic to the Lie group $(\lambda O_K,\tilde{+})$, where $\tilde{+}(z_1,z_2)\defeq F(+)(z_1,z_2)$ (see e.g. \cite[Proposition VII.2.2]{silverman}).
	
	It follows that, for $P \in E(K)$, denoting by $z(P)$ the $z$-coordinate in a minimal Weierstrass model, one has that:
	\begin{equation}\label{multiplesorder}
	z(n\sigma)=nz(\sigma)+O(z^2), \quad \forall n\in \Z.
	\end{equation}
	
	Moreover, for any $P \in E_1(K)$, we have that $\ord_s(P)=\ord_{\lambda}(z(P))$ (since $z$ is a local parameter for the $0$-section $O$). Hence, we have that $\ord_s(n\sigma)=\ord_{\lambda}(z(n\sigma))=\ord_{\lambda}(z(\sigma))=\ord_s(\sigma)$, where the middle equality follows from (\ref{multiplesorder}). Hence we have that $s_n=O(1)$.
	
	
	We prove now that $\delta_n$ is uniformly bounded (for any $n>0$). We notice that, in order for the intersection of $n \sigma$ and $O$ to be of order greater than $1$ at a certain point $t \in B$, the differential of (a local branch of) the Betti map $(\beta_1,\beta_2)\circ \sigma$ would have to be $0$ at the point $t$. As proven by P. Corvaja and U. Zannier in this paper, this happens only for finitely many base points $t$ in the base. We denote them by $t_1,\dots, t_k  \in B$. 
	
	For each $i=1,\dots, k$, let $\lambda_i$ denote a uniformizer for $t_i \in B(\C)$, and let $\rho^i_1, \rho^i_2$ denote a local choice (in a neighbourhood of $t_i$) of periods for the elliptic logarithm (see e.g. \cite[Section 1.1]{bettitorsion}). We then have then that (by standard intersection theory on complex surfaces, see e.g. \cite[Chapter I]{beauville}):
	\begin{equation}\label{multiplicity}
	\ord_{t_i}(n\sigma)=\dim_{\C} \faktor{\C\{\lambda_i\}}{(n\tilde{\sigma}(\lambda_i)-nb_1\rho^i_1(\lambda_i)-nb_2\rho^i_2(\lambda_i))}, 
	\end{equation}
	where $\C\{\lambda_i\}$ denotes the ring of locally analytic functions in the variable $\lambda_i$, $\tilde{\sigma}$ denotes the abelian logarithm of $\sigma$, and $b_1, b_2 \in \C$ are defined through the following condition: \[\tilde{\sigma}(t_i)=b_1\rho^i_1(t_i)+b_2\rho^i_2(t_i).\]	
	
	Since the ideal $(n\tilde{\sigma}(\lambda_i)-nb_1\rho^i_1(\lambda_i)-nb_2\rho^i_2(\lambda_i)) \subset \C\{\lambda\}$ does not depend on $n>0$, we have that the right hand side of (\ref{multiplicity}) does not depend on $n>0$, and we denote this quantity by $R_i$. As an immediate consequence of the above argument, we have that:
	\[
	\delta_n \leq R_1 + \dots + R_k - k,
	\] 
	from which we conclude that $\delta_n =O(1)=o(n^2)$, thus proving point $(1)$.
	
	We prove now point $(2)$. We use the fact that the Betti map is (locally) definable and bounded (by definable, we will always mean definable in $(\R_{an,exp})$). 
	
	Namely, it is proven in \cite[Section 10]{garethschmidt} that there exists a (definable) partition $B\setminus r^{-1}(S)=Y_1 \cup \dots Y_l$, and, on each $Y_j$ there exists a well defined branch of the Betti map, which we will denote by $B^j=(\beta^j_1,\beta^j_2)\circ \sigma: Y_j \rightarrow \R^2$, such that $B_j$ is definable, and in \cite[Proposition 4]{garethschmidt} that it is bounded. 
	
	Let us consider now, for each $j=1,\dots, l$, the following definable set in $\R^2 \times \R^2$:
	\[
	X_j\defeq \{(y,t) \in \R^2 \times \R^2: t \in Y_j \text{ and } y=B_j(t)\},
	\]
	i.e. $X_j$ is the transpose of the graph of $B_j$. Applying Hardt's Theorem (\cite[Theorem 9.1.2]{tametopology}) to $X_j$, we know that there exists a finite partition of $\R^2$, say $A_j^1\cup\dots \cup A_j^{d(j)}=\R^2$, such that $X_j$ is definably trivial over each $A_j^m$.  We recall that this means that, for each $m \leq d(j)$, there exists a definable set $F_j^m$, and a (definable) isomorphism $h_{A_j^m}:X_j\cap (A_j^m\times \R^2)\rightarrow A_j^m\times F_j^m$, that commutes with the projection to $A_j^m$. Moreover, since the Betti map is bounded on each $Y_j$, we may assume that, for each $m \leq d(j)$, either $A_j^m$ is bounded, or $X_j\cap (A_j^m\times \R^2)=\emptyset$. 
	
	We define now, for each $j=1,\dots, l$, $m \leq d(j)$:
	\begin{equation}
		f_j^m\defeq \#F_j^m
	\end{equation}
	
	We notice that $f_i^j$ is always finite. In fact, this is a direct consequence of the fact that the fibers of the Betti map $B_j$ are isolated points (see \cite[Proposition 1.1]{bettitorsion}).
	Then, the following equality holds:
	\begin{equation}\label{measure}
	\sum_{m \leq d(j)} f_{j}^m\mu(A_j^m)=\int_{B\setminus r^{-1}(S)} \sigma^*(\de \beta_1 \wedge \de \beta_2).
	\end{equation}
	
	In fact, notice that both left and right hand side of (\ref{measure}) are equal to the measure of the graph of the section $\sigma:\P_1(\C)\setminus S\rightarrow E\setminus \mathcal{S}_B$ %(which we denote, with a slight abuse of notation, by $\sigma \subset E\setminus \mathcal{S}$)
	, given by the integration of the restriction of the $(1,1)$-form $\de \beta_1 \wedge \de \beta_2 \in \Omega^2{E \setminus \mathcal{S}_B}$.
	%(which is, as we noticed earlier, indipendent of the choise of the branch of the Betti map, and, hence, well-defined on $E \setminus \mathcal{S}_B$).
	
	We notice now that, if $t \in \P_1(\C)\setminus S$, $\sigma(t) \in E_t$ is $n$-torsion if and only if the Betti coordinates $(\beta_1(\sigma(t)),\beta_2(\sigma(t)))$ are rational with denominator dividing $n$. Hence, the following equality holds, for each $n>0$:
	\begin{equation}\label{counting}
		A_n=\sum_{m \leq d(j)}f_{j}^m\alpha_{n,j}^m,
	\end{equation}
	where:
	\[
	\alpha_{n,j}^m\defeq \left\{\left(\frac{a}{n},\frac{b}{n}\right)\in A_j^m: \ a,b \in \Z\right\}.
	\]
	Hence:
	\[
		\frac{A_n}{n^2}=\sum_{m \leq d(j)}\frac{\alpha_{n,j}^m}{n^2},
	\]
	Letting $n \to \infty$, the thesis follows from (\ref{measure}), (\ref{counting}) and Lemma \ref{ominimalmeasure}.
\end{proof}

\begin{remark}\label{Rmk:torsion}
	In the proof of Theorem \ref{Prop:equality}, we treated separately the case of a torsion section $\sigma$. This was done because, in this case, the number $A_n$ (defined in \ref{targetorsion}) is not defined for all $n$. However, when $\sigma$ has order $m \neq 1$ (i.e. $\sigma \neq O$), the number $A_n$ is still defined for all $n$ coprime to $m$. Therefore, for such $\sigma$, one could use the same argument that works for a non-torsion section, changing the limits over $n$ into limits over $n$ coprime to $m$.
\end{remark}

\begin{remark}
	To prove Theorem \ref{Prop:equality}, one may also use a different argument. Namely, one first proves equality (\ref{soughtequality}) when $r:B \rightarrow \P_1$ is defined over $\bar{\Q}$ using a height argument, and then uses a continuity argument to prove it for every $r$. We give here a sketch of these two arguments. 
	
	
%	when $r:B \rightarrow \P_1$ is defined over a number field $K$, one may also use a different argument, which encompasses the use of o-minimal structures. We give here a sketch of said argument.
	
%	With the help of a specialization argument sketched below (in Note \ref{continuity}), one may then deduce equ 
	Assume that $r:B \rightarrow \P_1$ is defined over a number field $K$. One can prove in a straightforward way that: 
	\begin{equation}\label{Eq:approxsoughtequality}
	\lim_{n \to \infty}\frac{A^{\epsilon}_n}{n^2}=\int_{B \setminus r^{-1}(S_{\epsilon})} \sigma^*(\de \beta_1 \wedge \de \beta_2),
	\end{equation}
	where $S_{\epsilon}$ is a neighborhood of radius\footnote{With respect to some choice of a metric, which is irrelevant for our purposes. For instance, one may choose the Fubini-Study metric on $\P_1(\C)$.} $\epsilon$ of $S$, and 
	\[
	A^{\epsilon}_n=\# \{t \in B(\C)\setminus r^{-1}(S_{\epsilon}): \sigma(t) \text{ is }n-\text{torsion in } E_t\}.
	\]
	
	Now, one proves that:
	
	\begin{equation}\label{Bound:Uniform}
		\left| \frac{A^{\epsilon}_n}{A_n}-1 \right|\leq \frac{C}{\abs{\log \epsilon}}, \text{ where } C \in \R_+.
	\end{equation}
	
	
	Dividing $A_n$ in Galois orbits, this becomes a consequence of the fact that the points $t \in \cup_{n \in \N}A_n$ are of bounded height in $\P_1(\bar{\Q})$ \footnote{This is a direct consequence of the well-known result of Tate \cite{Tate} that, if $\pi:E\rightarrow B$ is an elliptic fibration, and $P:B \rightarrow E$ is a non-torsion section, then the function $t \mapsto \hat{h}_{E_t}(P_t)$, is, up to a bounded constant, a Weil height on $B$.}, 
	and the fact that points $x \in \P_1(\bar{\Q})$ of height $\leq C'$ have at most $\frac{C'}{\abs{\log \epsilon}}[K(x):K][K:\Q]$ conjugates of absolute value $< \epsilon$\footnote{One way to prove this is to use the following easy result, which can be found, for instance, in \cite[Remark 3.10(ii)]{LectureNotesZannier}. Let $\xi \in L$, where $L\subset \C$ is a Galois number field of degree $d$ over $\Q$, then:
	\[
	\sum_{g \in \Gal(L/\Q)}\abs{\log\abs{\xi^g}}\leq 2d\hat{h}(\xi).
	\]}.
	
	By letting $\epsilon \to 0$ in (\ref{Eq:approxsoughtequality}) and using (\ref{Bound:Uniform}) to employ a uniform convergence argument, one gets the sought equality (\ref{soughtequality}) for $r$.
	
	When $r$ is not defined over a number field, let $K_0 \subset \C$ denote its minimal field of definition, and let $K=K_0\bar{\Q} \subset \C$. The field $K$ has finite transcendence degree, hence there exists an integral algebraic variety $X/\bar{\Q}$ such that $\bar{\Q}(X)\cong K$. By construction, up to restricting $X$ to a Zariski open subset, we may assume that there exists an algebraic family of morphisms
	\[
	r_t:B_t \rightarrow \P_1, \ t \in X,
	\]
	and sections:
	\[
	\sigma_t:B_t \rightarrow E\times_{(\pi,r_t)}B_t, \ t \in X,
	\]
	such that there exists $t_0 \in X(\C)$, such that $B_{t_0}=B$, $r_{t_0}=r$ and $\sigma_t=\sigma_{t_0}$. 
	
	We want to show now that, for any $t \in X(\C)$ (and hence, in particular, for $t=t_0$), the following equality holds:
	\begin{equation}\label{soughtequalityt}
		\hat{h}(\sigma_t) = \int_{B_t \setminus r_t^{-1}(S)} \sigma^*(\de \beta_1^t \wedge \de \beta_2^t),
	\end{equation}
	
	where, for a point $P \in B_t$, $\beta_1^t$, $\beta_2^t$ denote the Betti coordinates of $\sigma_t(P)$.
	Now, we know that, for each $t \in X(\bar{\Q})$, (\ref{soughtequalityt}) is true by the argument presented above in this Remark. Since $X(\bar{\Q})$ is dense (in the euclidean topology) in $X(\C)$, it is hence sufficient to show that both right and left hand side of (\ref{soughtequalityt}) are continuous in $t \in X(\C)$. 
	
	For the left hand side, this is an immediate consequence of the fact that the height may be expressed through an explicit intersection formula (see \cite[Section 11.8]{ellipticsurfaces}), hence a standard flatness argument tells us that it is constant (hence, continuous) in $t$.
	
	For the right hand side, this follows from a dominated convergence argument, which proceeds as follows. Since the $(1,1)$-form $\de \beta_1 \wedge \de \beta_2$ diverges only when $\lambda$ approaches $0,1$ or $\infty$, it is sufficient to give a bound for $\de \beta_1 \wedge \de \beta_2$ in small disks around these three points, locally uniformly in $t \in X(\C)$. Without loss of generality, one may do so only for a small disk around $0$. Here, one may use results of Jones and Schmidt \cite{garethschmidt} to obtain the following bound in a circle $\abs{\lambda} < \epsilon$, locally uniformly in $t \in X(\C)$:
	\begin{equation}\label{bound}
	\de \beta_1^t \wedge \de \beta_2^t= \bigO\left(\frac{1}{\abs{\lambda}^2 \abs{\log \lambda}^4}\right)\de \lambda \wedge \de \bar{\lambda}. 
	\end{equation}
	Since the right hand side of (\ref{bound}) is an integrable $2$-form in the circle $\abs{\lambda}<\epsilon$, this allows a dominated convergence argument to be employed to show that the right hand side of (\ref{soughtequalityt}) is continuous, hence reproving Theorem \ref{Prop:equality}.
\end{remark}
%
%
%\begin{note}\label{continuity}
%{\color{red}Questo argomento è solo una bozza che ho appena aggiunto}
%
%
%	When $r:B \rightarrow \P_1$ is not algebraic, . The argument might go as follows. Let $r:B \rightarrow \P_1$ be a finite morphism, defined over $\C$, where $B$ is a smooth complex curve. By a standard specialization argument, one may prove that any such morphism belongs to an (algebraic and flat) family $r_t:B_t\rightarrow \P_1$, $t \in X$, where $X/\bar{\Q}$ is an algebraic variety. Let $t_0$ be such that $r_{t_0}=r$. We may assume without loss of generality that $X$ is smooth. If one has a section $\sigma:B \rightarrow E$, one may extend it (again, by a specialization argument) to an (algebraic) family of sections $\sigma_t:B_t \rightarrow E$, fot $t$ in an étale neighbourhood  $Y \rightarrow X$ of $t_0$. Now, both the canonical height $\hat{h}(\sigma)$ and the integral $\int_{B_t \setminus r_t^{-1}(S)} \sigma_t^*(\de \beta_1 \wedge \de \beta_2)$ should be\footnote{For the height, it should be clear that it is not only continous, but (locally) constant in $t$. In fact, each of the intersection numbers $<n\sigma_t,O_t>$ should be constant in a neighbourhood of $t_0$, because of the flatness over $X$. For the integral, one needs to take care of the singularities. Most likely, this can be done with the estimates on the differential of the Betti map.}... continous functions in $t$. Hence, one concludes using euclidean density of $\bar{\Q}$-points in $X$.
%\end{note}

\begin{corollary}
	Let $\sigma:B \rightarrow E_B$ denote a section of the elliptic surface $E_B \rightarrow B$. Then the integral $\int_{B \setminus r^{-1}(S)} \sigma^*(\de \beta_1 \wedge \de \beta_2)$ has a rational value.
\end{corollary}
\begin{proof}
	We have by \cite[Section 11.8]{ellipticsurfaces} that $\hat{h}(\sigma) \in \Q$. Hence, the thesis is an immediate consequence of Theorem \ref{Prop:equality}.
\end{proof}




We give here another expression for the $(1,1)$-form $\de \beta_1 \wedge \de \beta_2 \in \Omega^2(E \setminus \mathcal{S})$ that might be more suitable for calculations. To fix some notation, let $\Lambda_{\lambda}$ be the unique lattice in $\C$ corresponding to the elliptic curve $E_{\lambda}$ with Weierstrass form $y^2=x(x-1)(x-\lambda)$ (see Remark \ref{Rmk:weierstrass}), and let $\rho_1(\lambda),\rho_2(\lambda)$ be a local choice of periods. Moreover, let $d(\lambda)\defeq\rho_1(\lambda)\overline{\rho_2(\lambda)}-\rho_2(\lambda)\overline{\rho_1(\lambda)}=2iV(\lambda)$, where $V(\lambda)$ denotes the (oriented) area of the fundamental domain of $\Lambda_{\lambda}$. Let $\eta_{\lambda}(\zeta)$ denote the linear extension to $\C$ of the quasi-period function $\eta_{\lambda}$ (which is defined on the lattice $\Lambda_{\lambda}$ as in \cite[VI.3.1]{silverman1994advanced}). Then, the following equality holds (here we are denoting, with a slight abuse of notation, by $\beta_i(\lambda)$ the Betti coordinates $\beta_i(\sigma(\lambda))$:
\begin{equation}\label{riscrittura}
	\de \beta_1(\lambda) \wedge \de \beta_2(\lambda)=\frac{1}{d(\lambda)}\left(\de z \wedge \de \bar{z}+\frac{1}{2\lambda}(\eta \de \lambda \wedge \de \bar{z}+\bar{\eta}\de{z}\wedge\de{\bar{\lambda}})+\frac{1}{4\lambda^2}\de \lambda \wedge \de \bar{\lambda}\right).
\end{equation}



\begin{example}\label{explicitexample}
	Let us do an explicit computation of the terms of \ref{soughtequality}, in the case of a specific section, for instance:
	\[
	\sigma(\lambda)=(2,\sqrt{2(2-\lambda)}).
	\]
	This section is not defined over the base curve $\P_1$. It is, however, well defined as a section of the elliptic surface $E_B \defeq E \times_{(\pi,\phi)}B \rightarrow B$, where $B \cong \P_1$, and $\phi(t)=t^2+2$. 
	
	One can compute $\hat{h}(\sigma)$ explicitly, by using the intersection product on a proper regular model of $E_B$ (see \cite[Section 11.8]{ellipticsurfaces})\footnote{Our normalization of the height function differs by a multiplicative factor of $\frac{1}{2}$ from that of \cite{ellipticsurfaces}.}. A simple application of Tate's algorithm reveals that the fibration $\pi:E_B \rightarrow B$ has $5$ singular fibers, four of which are of type $I_2$ (the ones over $\lambda= 0,1$) and one of type $I_4$ (the one over $\lambda=\infty$). Looking at the intersection of the section $\sigma$ with the singular fibers reveals that the canonical height $\hat{h}(\sigma)=\frac{11}{8}$.
	
	Using (\ref{riscrittura}), the $(1,1)$-form $\sigma^*(\de \beta_1 \wedge \de \beta_2)$ is equal to the following:
	\begin{gather}\label{Expr:11formexample}
	\de \beta_1(\lambda) \wedge \de \beta_2(\lambda)=\\ \frac{1}{d(\lambda)}\left(\de z(\lambda) \wedge \de \overline{z(\lambda)}+\frac{1}{2\lambda}(\eta \de \lambda \wedge \de \overline{z(\lambda)}+\overline{\eta}\de{z(\lambda)}\wedge\de{\overline{\lambda}})+\frac{1}{4\lambda^2}\de \lambda \wedge \de \overline{\lambda}\right).
	\end{gather}
	Here:
	\begin{equation}
		z(\lambda)=\int_{\infty}^{2}dx/\sqrt{x(x-1)(x-\lambda)}
	\end{equation}
	(the determination chosen for the $\sqrt{}$ is irrelevant for our purposes). 
	Hence, we deduce the following integral identity from Theorem \ref{Prop:equality}:
	\[
	\int_{\lambda \in \C} \de \beta_1(\lambda) \wedge \de \beta_2(\lambda) = \frac{1}{2}	\int_{B} \de (\beta_1\circ \sigma) \wedge \de (\beta_2\circ \sigma)=\frac{1}{2}\cdot \frac{11}{8}=\frac{11}{16}.
	\]
\end{example}

\paragraph{Comparison with a measure coming from dynamics}

Let us cite the following result from \cite[Section 3]{demarcomavraki}:
\begin{theorem}[DeMarco,Mavraki]\label{Theo:demarcomavraki}
	Let $\pi:E \rightarrow B$ be an elliptic surface and $P : B \rightarrow E$ a non-torsion section, both defined over $\Q$. Let $S \subset E$ be the union of the finitely many singular fibers in $E$. There is a positive, closed $(1,1)$-current $T$ on $E(\C) \setminus S$ with locally continuous potentials such that $\restricts{T}{E_t}$ is the Haar measure on each smooth fiber, and $P^*T$ is equal to a measure $\mu_{P}$, that satisfy the following property. For any infinite non-repeating sequence of $t_n \in B(\bar{\Q})$, such that $\hat{h}_{E_{t_n}}(P_{t_n}) \rightarrow 0$ as $n \to \infty$, the discrete measures
	\[
	\frac{1}{\#\Gal(\bar{\Q}/\Q)t_n}\sum_{t \in \Gal(\bar{\Q}/\Q)t_n}\delta_{t_n}
	\]
	converge weakly on $B(\C)$ to $\mu_P$.
\end{theorem}



The restriction of the current $T$ to the open set $E \setminus (\mathcal{S} \cup O)$ (where we are denoting by $O$, with a slight abuse of notation, the image of the zero section of $\pi:E \rightarrow B$), can be written down explicitly as follows (\cite[Section 3.3]{demarcomavraki}):
\begin{equation}\label{Expr:11current}
	T=\frac{1}{2\pi i}\de \de^c {H_N},
\end{equation}
where $H_N$ denotes the Néron local (archimedean) height function. We recall that the following formula holds \cite[p. 466]{silverman1994advanced}:
\begin{equation}\label{Expr:altezzalocale}
	H_N=-\log \abs{e^{-\frac{1}{2}z\eta_{\lambda}(z)}\sigma_{\lambda}(z)\Delta(\Lambda_{\lambda})^{\frac{1}{12}}}.
\end{equation}
Here $\Lambda_{\lambda}$ denotes a lattice in $\C$ such that $\C/\Lambda_{\lambda} \cong E_{\lambda} \defeq \pi^{-1}(\lambda)$, $z$ denotes the complex variable of $\C/\Lambda_{\lambda}$ and $\eta_{\lambda}$ and $\sigma_{\lambda}$ indicate the elliptic functions $\eta$ and, resp., $\sigma$ (as defined in \cite[VI.3.1,I.5.4]{silverman1994advanced}) associated to the lattice $\Lambda_{\lambda}$. Since $\sigma_{\lambda}(z)\Delta(\Lambda_{\lambda})^{\frac{1}{12}}$ is a holomorphic function in both variables $\lambda$ and $z$, this gives the following expression for $T$:
\begin{equation}\label{Expr:11current2}
	T=\frac{1}{4\pi i}\de \de^c(\Re (z\eta_{\lambda}(z))).
\end{equation}

%The current $T$ restricts to the normalized Haar measure on the smooth fibers of $\pi$ (see e.g. \cite{...}).

We shall now prove that the current $T$ is equal to $(1,1)$-current defined by the $(1,1)$-form $\de \beta_1 \wedge \de \beta_2$ on $E \setminus \mathcal{S}$. We shall prove this in two different ways: through a direct calculation (we give a sketch in Remark \ref{Rmk:calculationforT}) and through a dynamical argument (in Corollary \ref{Cor:dynamicalforT}).

We note that both $T$ and $\de \beta_1 \wedge \de \beta_2$ restrict to the Haar measure on the fibers, normalized in such a way that the area of each fiber is $1$. 
%This can be checked with a direct calculation, as both forms are expressed explicitly. 



\begin{remark}\label{Rmk:calculationforT}
	Let us give a sketch of a calculation that shows that $T= \de \beta_1 \wedge \de \beta_2$. One first checks that:
	\begin{equation}\label{Expr:Traw}
	T=\frac{1}{4\pi i}\partial \bar{\partial} \left[ \begin{pmatrix}
	z \\ \bar{z}
	\end{pmatrix}^\intercal
	\begin{pmatrix}
	\eta_1 & \eta_2 \\
	\bar{\eta_1} & \bar{\eta_2}
	\end{pmatrix}
	\begin{pmatrix}
	\rho_1 & \rho_2 \\
	\bar{\rho_1} & \bar{\rho_2}
	\end{pmatrix}^{-1}
	\begin{pmatrix}
	z \\ \bar{z}
	\end{pmatrix}
	\right],
	\end{equation}
	\begin{equation}\label{Beta}
	\begin{pmatrix}
	\de \beta_1 \\ \de \beta_2
	\end{pmatrix}=
	A^{-1}\left(\de(A)A^{-1}\begin{pmatrix}
	z \\ \bar{z}
	\end{pmatrix}+\begin{pmatrix}
	\de z \\ \de \bar{z}
	\end{pmatrix}\right),  \text{ where } \ A\defeq \begin{pmatrix}
	\rho_1 & \rho_2 \\
	\bar{\rho_1} & \bar{\rho_2}
	\end{pmatrix}.
	\end{equation}
	Let $\gamma_{\lambda}$ be the $1$-form $\de \lambda/2\lambda$. Using the relations $\frac{\de \rho_i}{\de \lambda}=\frac{1}{2\lambda} \eta_i$, we see that:
	\[
	\de (A)A^{-1}= \begin{pmatrix}
	\gamma_{\lambda} & 0 \\ 0 & \overline{\gamma_{\lambda}}
	\end{pmatrix}\begin{pmatrix}
	\eta_1 & \eta_2 \\
	\bar{\eta_1} & \bar{\eta_2}
	\end{pmatrix}
	\begin{pmatrix}
	\rho_1 & \rho_2 \\
	\bar{\rho_1} & \bar{\rho_2}
	\end{pmatrix}^{-1}.
	\] 
	To simplify the notation, we define:
	\[ 
	C\defeq \begin{pmatrix}
	\ C_1 \ \ \\  \ C_2 \ \
	\end{pmatrix}\defeq \begin{pmatrix}
	C_{11} & C_{12} \\ C_{21} & C_{22}
	\end{pmatrix}
	\defeq \begin{pmatrix}
	\eta_1 & \eta_2 \\
	\bar{\eta_1} & \bar{\eta_2}
	\end{pmatrix}
	\begin{pmatrix}
	\rho_1 & \rho_2 \\
	\bar{\rho_1} & \bar{\rho_2}
	\end{pmatrix}^{-1}
	\]
	\[
	=\frac{1}{\det (A)}
	\begin{pmatrix}
	\eta_1 \bar{\rho_2}-\eta_2\bar{\rho_1} & 2\pi i \\ 2 \pi i& -\bar{\eta_1}\rho_2+\bar{\eta_2}\rho_1
	\end{pmatrix}.\footnote{We have used here the Legendre relation $\rho_2\eta_1-\rho_1\eta_2=2 \pi i$}
	\]
	We wish now to express both $\de \beta_1 \wedge \de \beta_2$ and $T$ as linear combinations (with coefficients in $\mathcal{C}^{(\infty)}(\pi^{-1}(U))$, $U \subset \P_1 \setminus \{0,1,\infty\}$ being an open simply connected domain) of the forms $\gamma_{\lambda} \wedge \overline{\gamma_{\lambda}}, \gamma_{\lambda} \wedge \de \bar{z},\de z \wedge \overline{\gamma_{\lambda}},\de z \wedge \de \bar{z}$. We note that a (smooth) $(1,1)$-form in $\Omega^{1,1}(\pi^{-1}(U))$ can be written uniquely as a linear combination of these forms, since the two $1$-forms $\gamma_{\lambda}$ and $\de z$ span the cotangent bundle over each point in $\pi^{-1}(U)$. Hence it will suffice to check that the two expressions for $\de \beta_1 \wedge \de \beta_2$ and $T$ are the same.
	
	We first do this for $\de \beta_1 \wedge \de \beta_2$.
	
	Namely, using (\ref{Beta}), one deduces that:
	\begin{equation}\label{Betaform}
	\de \beta_1 \wedge \de \beta_2=\frac{1}{\det (A)}\left[(C_{11}z+C_{12}\bar{z}){\gamma_{\lambda}}+\de z\right]\wedge\left[(C_{21}z+C_{22}\bar{z}){\overline{\gamma_{\lambda}}}+\de \bar{z}\right].
	\end{equation}
	
	We turn now to $T$.
		
	One can verify that:
	\[
	\partial C_2=\frac{-2\pi i{\gamma_{\lambda}}}{\det(A)}C_1, \quad \bar{\partial} C_1=\frac{-2\pi i\overline{\gamma_{\lambda}}}{\det(A)}C_2,
	\]
	\[
	\partial \bar{\partial} C_1=\frac{2\pi i}{\det (A)}[-C_{21}C_1-C_{11}C_2]\gamma_{\lambda}\wedge \overline{\gamma_{\lambda}}, 
	\]
	\[  \bar{\partial} \partial C_2=\frac{4\pi i}{\det (A)}[C_{12}C_2+C_{22}C_1]\gamma_{\lambda}\wedge \overline{\gamma_{\lambda}}.
	\]
	
	Using the Leibniz rule in (\ref{Expr:Traw}), and the above expressions, one gets:
	\[
	T=\frac{1}{4\pi i}\cdot \frac{1}{\det (A)}\left[ 4 \pi i\de z \wedge \de \bar{z}+2 \begin{pmatrix}
	\de z \\ \de \bar{z}
	\end{pmatrix}\wedge \begin{pmatrix}
	\  - 2 \pi iC_2 \overline{\gamma_{\lambda}} \ \ \\  \ { 2 \pi i }C_1 \gamma_{\lambda} \ \
	\end{pmatrix}\begin{pmatrix}
	 z \\  \bar{z}
	\end{pmatrix}\right.
	\]
	\begin{equation}\label{Expr:T}
		\left.+\frac{2 \pi i}{\det (A)}\begin{pmatrix}
		z \\ \bar{z}
		\end{pmatrix}\begin{pmatrix}
		\ C_{21}C_1+C_{11}C_2 \ \  \\ C_{12}C_2+C_{22}C_1
		\end{pmatrix}\begin{pmatrix}
		z \\ \bar{z}
		\end{pmatrix}\gamma_{\lambda}\wedge \overline{\gamma_{\lambda}}\right].
	\end{equation}
	
	An easy term-by-term comparison reveals that the expressions (\ref{Expr:T}) and (\ref{Betaform}) are the same.
\end{remark}


We focus now on giving a more theoretical proof of the equality $T=\de \beta_1 \wedge \de \beta_2$.


\begin{proposition}\label{Prop:uniqueform}
	Let $D \subset \P_1\setminus \{0,1,\infty\}$ be an open simply connected domain. The $(1,1)$-form $\de \beta_1 \wedge \de \beta_2$, is, up to scalar multiplication, the unique closed\footnote{If one desires to, one can replace the \textit{closed} hypothesis with the hypothesis that the restriction of $\omega$ to the fibers of $\pi$ is the normalized Haar measure. We note that the two different hypothesis are logically independent among each other. On the other hand, the proof that is presented here works in both cases.} $2$-form in $\Omega_0^2(\pi^{-1}(D))$\footnote{We are using the notation $\Omega_0^2$ to denote continuous $2$-forms. } that satisfies $[2]^*\omega=4\omega$, where  $[2]:\pi^{-1}(D) \rightarrow \pi^{-1}(D)$ denotes the endomorphism of multiplication by $2$ on the fibers.
\end{proposition}
\begin{proof}
	Let $\omega \in \Omega_0^2(\pi^{-1}(D))$ be a $2$-form such that $[2]^*\omega=4 \omega$.
	
	Let:
	\[
	\omega=\sum_{i,j}\alpha_{ij} \gamma_i\wedge \gamma_j, %\alpha_{12}\de \beta_1 \wedge \de \beta_2 + \alpha_{1\lambda} \de \beta_1 \wedge \de \lambda +\alpha_{1\lambda} \de \beta_1 \wedge \de \bar{\lambda}+\alpha_{1\lambda} \de \beta_2 \wedge \de \lambda+\alpha_{1\lambda} \de \beta_2 \wedge \de \bar{\lambda}+\alpha_{\bar{\lambda}\lambda} \de \bar{\lambda} \wedge \de \lambda.
	\quad \text{where } i,j \in \{1,2,\lambda,\bar{\lambda}\} \text{, } \ \begin{cases}
	\gamma_i=\de \beta_i \ \ i=1,2 \\
	\gamma_i=\de \lambda \  \ i=\lambda \\
	\gamma_i= \de \bar{\lambda} \ \ i=\bar{\lambda}
	\end{cases},
	\]
	and $\alpha_{ij} \in \mathcal{C}^0(\pi^{-1}(D))$.
	
%	Since the thesis is local in $D$, we may prove it by restricting to a simply connected subdomain $D' \subset D$, such that $\overline{D'}\subset D$ is compact. The open  set $\pi^{-1}(D')$ is then compactly closed in $\pi^{-1}(D)$ as well. 
	
	We notice that: 
	\[
	\frac{1}{4}[2]^*\omega=\sum_{ij}\alpha_{ij}([2]P) 2^{\delta{i}+\delta{j}-2}\gamma_i \wedge \gamma_j,
	\]
	where $\delta_i=1$ when $i=1,2$, and $\delta_i=0$ otherwise.
	We know that $\frac{1}{4}[2]^*\omega=\omega$, and, hence:
	\begin{equation}\label{Eq:equivariantforms}
	\alpha_{ij}(P)=2^{\delta{i}+\delta{j}-2}\alpha_{ij}([2]P), \quad \text{for all } i,j \in \{1,2,\lambda,\bar{\lambda}\} \text{ and } P \in \pi^{-1}(D').
	\end{equation}
	However, for each $x \in D$, restricting both hand sides of equation (\ref{Eq:equivariantforms}) to the fiber $\pi^{-1}(x)$, and then taking the $\infty$-norm yields:
	\[
	\max_{P \in \pi^{-1}(x)} \abs{\alpha_{ij}(P)}=2^{\delta{i}+\delta{j}-2}\max_{P \in \pi^{-1}(x)} \abs{\alpha_{ij}(P)}\leq 2^{-1}\max_{P \in \pi^{-1}(x)} \abs{\alpha_{ij}(P)} \ \text{if } \{i,j\}\neq \{1,2\}.
	\]
	Hence, for $\{i,j\}\neq \{1,2\}$, $\alpha_{ij}\equiv 0$. Therefore, we have that:
	\[
	\omega=\alpha_{12}\de \beta_1 \wedge \de \beta_2,
	\]
	\[
	\alpha_{12}([2]P)=\alpha_{12}(P) \quad \forall P \in \pi^{-1}(D').
	\]
	
	Since the function $\alpha_{12}$ is continous, this implies that $\alpha_{12}$ is constant on the fibers of $\pi$, and hence it depends just on $\lambda$ and $\bar{\lambda}$. Since $\omega$ is closed by hypothesis, it follows that $\alpha_{12}$ is constant in $\lambda$ and $\bar{\lambda}$ as well
	%\footnote{If we chose to substitute the \textit{closed} hypothesis with the hypothesis that $\omega$ restricts to the normalized Haar measure on the fibers, $\alpha_{12}$ is automatically constant, and equal to $1$.}
	. Hence $\omega = c\de \beta_1 \wedge \de \beta_2$, with $c \in \C$.
\end{proof}

\begin{remark}\label{Rmk:hypothesis}
	In Proposition \ref{Prop:uniqueform}, one can replace the \textit{closed} hypothesis with the hypothesis that the restriction of $\omega$ to the fibers of $\pi$ is the normalized Haar measure. We note that the two different hypothesis do not \textit{a priori} imply each other. On the other hand, the proof that was presented here works in both cases. In fact, the only point in which we used the closed assumption was to deduce that the function $\alpha_{12}$ is constant. However, this is automatic if $\omega$ restricts to the normalized Haar measure on the fibers.
\end{remark}

\begin{remark}
	In Proposition \ref{Prop:uniqueform}, the hypothesis that $\omega$ is continous on $\pi^{-1}(D)$ is crucial. The $(1,1)$-form $\de z \wedge \de \bar{z}$, which is not a constant multiple of $\de \beta_1 \wedge \de \beta_2$, satisfies all the hypothesis of the proposition, except that it is not a well-defined continous $2$-form on $\pi^{-1}(D)$. This last fact may be easily seen by noticing that summing a period $\rho(\lambda)$ to $z$ changes the $2$-form $\de z \wedge \de \bar{z}$ by a (non-zero) term $\rho'(\lambda)\overline{\rho'(\lambda)}\de \lambda \wedge \de \bar{\lambda}$. 
\end{remark}


\begin{corollary}\label{Cor:dynamicalforT}
	The restriction to $E \setminus \mathcal{S}$ of the $(1,1)$-current $T=\frac{1}{2\pi i}\de \de^c {H_N}$ is equal to the current associated to the $(1,1)$-form $\de \beta_1 \wedge \de \beta_2$.
\end{corollary}
\begin{proof}
	Let $D \subset \P_1\setminus S$ be an open relatively compact simply connected domain. We notice that $T=\frac{1}{2\pi i}\de \de^c {H_N}$ is obviously closed. Moreover, we have that:
	\[
	([2]^*T)(P)=\frac{1}{2\pi i}\de \de^c (H_N([2]P)), \ \forall\  P \in \pi^{-1}(D).
	\]
	We have the following well-known equality (see e.g. \cite[Theorem VI.1.1]{silverman1994advanced}):
	\[
	H_N([2]P)=4H_N(P)-\log \abs{2y}+\frac{1}{4}\log \abs{\Delta_\lambda},
	\]
	where a Weierstrass form for the elliptic curve $E_\lambda$\footnote{Since we are working in characteristic $0$, we are assuming here that the Weierstrass form is in the shape:
	\[
	y^2=x^3+bx^2+cx+d.
	\]} is assumed to have been fixed. Since we are working in the Legendre family, we may, of course, choose the Legendre form.
	
	Since $\de \de^c \log \abs{f}=0$, for any holomorphic function $f$, this implies that:
	\[
	([2]^*T)(P)=\frac{1}{2\pi i}\de \de^c (H_N([2]P)))=4\frac{1}{2\pi i}\de \de^c H_N(P)=4T, \ \forall\  P \in \pi^{-1}(D).
	\]
	
	We prove now that $\restricts{T}{E\setminus \mathcal{S}}$ is smooth. We already know that $T$ is smooth away from the zero-section $O$. Let $E_2:D \rightarrow E$ denote a section of $\pi$ of order $2$ (i.e. $[2]E_2=O$, $E_2 \neq O$). By restricting the equality $([2]^*T)(P)=4T(P)$ to a neighborhood of the section $E_2$, we get a smooth $(1,1)$-current on the right hand side, hence the left hand side has to be smooth as well. Since the map $[2]:E\setminus \mathcal{S}  \rightarrow E\setminus \mathcal{S}$ defines a biholomorphism between a neighborhood of $E_2$ and a neighborhood of $O$ (since we are restricting to the good reduction), this in turn tells us that the restriction of $T$ to a neighborhood of $O$ is smooth. Hence the restriction of $T$ to $E \setminus \mathcal{S}$ is smooth, and hence is represented by a $(1,1)$-form, which, with a slight abuse of notation, we will still denote by $T$.
	
	Hence, by Proposition \ref{Prop:uniqueform}\footnote{Here, instead of using Proposition \ref{Prop:uniqueform}, we could use its modified version, as in Remark \ref{Rmk:hypothesis}.}, $T=c \de \beta_1 \wedge \de \beta_2$, where $c \in \C$ is a constant. Since, as remarked at the beginning of this section, both $T$ and $\de \beta_1 \wedge \de \beta_2$ restrict to the normalized Haar measure on the fibers of $\pi$, $c=1$, as we wanted to prove.
\end{proof}



\bibliographystyle{alpha}
\bibliography{Betti}


\end{document}