...

%%	
%%	To do so, by dividing $S_{\epsilon}$ in at most $3 \deg r$ disks of radius $\epsilon$, it is sufficient to prove the following:
%%	\[
%%	\#\{t \in A_n:\abs{t}< \epsilon \}\leq \frac{C}{\log \epsilon} \# A_n.
%%	\]
%%	
%%	
%%	
%%	
%%	To do so, one uses the following well-known bound on the points $t \in A_n$:
%%	\[
%%	...
%%	\]
%%	
%%	To do so, we divide $A_n=\cup_{t^n_i \in \P_1(\bar{\Q})\Gal(\bar{\Q}/\Q)t^n_i$ in Galois orbits. Now, in each Galois orbit the proportion of points inside $S_{\epsilon}$ is limited by $\frac{C}{\log \epsilon}$.
%	
%	To do so, one observes that the points $t \in \cup_{n \in \N}(A_n)$ are of bounded height (this is a direct consequence of the well-known result of Tate \cite{...}), and then uses bounds on the number of points of bounded height in a small disk \footnote{Namely, we use the following bound: the points in $\bar{\Q}$ of height $\leq C$ }.

%\begin{equation}\label{riscrittura}
%	\de b_1 \wedge \de b_2=\frac{1}{d}(\de z \wedge \de \bar{z}+\beta_1^2 \de \rho_1 \wedge \de \bar{\rho_1}
%										+\beta_1\beta_2 \de \rho_1 \wedge \de \bar{\rho_2}+\beta_1\beta_2 \de \rho_2 \wedge \de \bar{\rho_1}
%										+\beta_2^2 \de \rho_2 \wedge \de \bar{\rho_2}+)
%\end{equation}


...
Let $\rho_1$ and $\rho_2$ be a choice of local periods, and let $\eta_1$, $\eta_2$ denote the corresponding quasi-periods. Suppose that these are ordered in such a way that:
\[
\rho_2\eta_1-\rho_1\eta_2=2\pi i.
\]
Let $z$ be a (local branch) of the abelian logarithm, and let $\lambda$ denote the coordinate on the base of the Legendre scheme. Following the notation of \ref{...}, we denote the (local branch of) the Betti coordinates by $(\beta_1,\beta_2)$.

We note that the following relations hold:
\[
\begin{pmatrix}
z \\ \bar{z}
\end{pmatrix}
=
\begin{pmatrix}
\rho_1 & \rho_2 \\
\bar{\rho_1} & \bar{\rho_2}
\end{pmatrix}
\begin{pmatrix}
\beta_1 & \beta_2
\end{pmatrix},
\]

\[
\begin{pmatrix}
\eta \\ \bar{\eta}
\end{pmatrix}
=
\begin{pmatrix}
\eta_1 & \eta_2 \\
\bar{\eta_1} & \bar{\eta_2}
\end{pmatrix}
\begin{pmatrix}
\beta_1 & \beta_2
\end{pmatrix}.
\]
It will be convenient to use the following notations:
\begin{equation*}
\mathbf{z}\defeq \begin{pmatrix}
z \\ \bar{z}
\end{pmatrix},
\quad
	A\defeq \begin{pmatrix}
\rho_1 & \rho_2 \\
\bar{\rho_1} & \bar{\rho_2}
\end{pmatrix}, 
\quad
	 B\defeq \begin{pmatrix}
\eta_1 & \eta_2 \\
\bar{\eta_1} & \bar{\eta_2}
\end{pmatrix},
\quad
T\defeq BA^{-1}.
\end{equation*}
\begin{equation*}
	T=\frac{1}{\det A}\begin{pmatrix}
	.. & .. \\ .. & ..
	\end{pmatrix}=\begin{pmatrix}
	\ \ T_1 \ \ \\ \ \ T_2 \ \
	\end{pmatrix}.
\end{equation*}
The following relations hold:
...



...
If one chooses local coordinates $z, \lambda$ on the surface $E$, such that $\lambda$ is just a (local) coordinate on the base, and $z$ is the complex coordinate on the fiber (i.e. the value, up to periods, of the elliptic integral on the fibers), then one can (locally) decompose every $(1,1)$-form in the following way:
\[
\omega=\alpha_{z,\bar{z}}\de z \wedge \de \bar{z} +\alpha_{\bar{z},\lambda}\de \bar{z} \wedge \de \lambda+\alpha_{z,\bar{\lambda}}\de z \wedge \de \bar{\lambda}+\alpha_{\lambda,\bar{\lambda}}\de \lambda \wedge \de \bar{\lambda}.
\]
If we let $\omega$ be $(1,1)$-form $T- \de \beta_1 \wedge \de \beta_2$, we know that the coefficient $\alpha_{z,\bar{z}}$ is $0$ (since it is so when restricted to the fibers). One might expect that the remaining pieces are, for instance, not closed for the operator $\de$ or $\de^c$ (while, $T- \de \beta_1 \wedge \de \beta_2$ is so).
...

...
